\section{Introduction}
[...]

\section{Making sense of the data: descriptive statistics and static visualizations}

We have scraped the responses from the Danish politicians to the voting advice application (VAA) at [DR's homepage][1]. 724 candidates answered the VAA out of a total of 799 candidates running ([source][2]) and thus accounts for 90,6\% of all candidates and 1.630.587 personal votes out of 1.762.656 given. This difference in personal votes can to a large extent be attributed to the fact that two opposing party leaders, Lars Løkke Rasmussen and Helle Thorning-Schmidt, did not partake in the VAA. The 1.630.587 personal votes accounts for 46,3\% of all (valid) votes (3.518.987 in total). 
There are five candidates who are not running for a particular party (independents) and these are discarded since they are very few, they do not receive a remarkable amount of votes and they have no party affiliation. 
The candidates are asked to rank 15 questions on political issues on a "Highly disagree-Mostly disagree-Neither agree nor disagree-Mostly agree-Highly agree" scale. The questions vary from tariffs on cigarettes over public sector growth to the amount of influence given to EU. The questions are all weighted equally and all 724 candidates have answered all questions. The purpose of the VAA is to guide the voters in the jungle of candidates. By partaking in the VAA the voter is able to see which candidate she agrees with the most and hence, might help her decide on who to vote for. Numerous studies have been made in order to clarify the power of these VAAs on voting behavior, mainly from survey studies. (referencer!) An obvious further extension to this paper is to collect the unique answers from the users (the voters) to see if there is a correlation with the election outcome.  
Furthermore we have scraped additional descriptive data on the candidates such as gender, age, current position, whether they ran for office at the previous election and whether they were elected at this election or not. When using data from the internet it is important to consider the ethical issues in using data that the candidates might not be aware of. Given that these candidates are openly running for the parliamentary election and that the main webpage from which we have been scraping is the homepage of the public service national broadcasting corporation, DR, we conclude that we are not violating any ethical conducts. 

[Figur 1 - mean respone, parties]

We start by computing an overview of the mean responses to question on a party-level. (Appendix??) First of all we see that the mean responses to all questions seem rather spread out in the sense that there is no immediate clusterings around one end of any questions. Second of all we see a strong tendency for a pattern at least for the orange and turqoise dots (Enhedslisten and Liberal Alliance) that seem to linger around the (opposing) edges of the mean response spectrum and rarely around the middle. There are also some dots that are almost completely, or to a large extent, overlapping - as is often the case for Enhedslisten (orange) and SF (pink), which makes sense politically. However we also see opposing parties joining views in matters of EU (Enhedslisten and DF) and the public school reform (Enhedslisten and Liberal Alliance).
These observations are validated even further when we compute the average distance from the "neither agree nor disagree" for each party (figure 2), where Enhedslisten and Liberal Alliance, not surprisingly cf. figure 1, are distinctively more extreme in their opinions. The average distance for the most extreme (Enhedslisten) is almost twice as large as the least extreme (Kristendemokraterne). 

